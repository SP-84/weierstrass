\documentclass{amsart}
\usepackage[T1]{fontenc} % French fontenc
\usepackage[french]{babel} % Use French as default lang
\usepackage{mathtools}
\usepackage{amsfonts}
\usepackage[backend=biber]{biblatex} % Simple bibliography usage with biber
\addbibresource{weierstrass.bib}

\newtheorem{definition}[subsubsection]{Définition}
\newtheorem{theorem}[subsubsection]{Théorème}
\newtheorem{corollary}[subsubsection]{Corollaire}
\newtheorem{lemma}[subsubsection]{Lemme}

\title{AMS Théorème de Weierstrass et variations}
\author{Guillaume Salloum}
%\email{guillaume.salloum@alumni-univ-avignon.fr}

\begin{document}
\begin{abstract}
	Nous nous intéresserons au théorème d'approximation de Weierstrass et
	à sa présentation sous diverses formes. Dans un premier temps on posera le cadre
	général avec la version complexe de Stone-Weierstrass.
	Ensuite on reprendra le cas
	des séries de Fourier obtenu à l'aide du théorème de Fejer vu en cours. 
	Puis on s'intéressera aussi au résultat sur les polynômes de Bernstein permettant
	de fournir une autre preuve constructive du théorème dans le cas réel.
\end{abstract}

\maketitle
\setcounter{tocdepth}{2}
\tableofcontents

\section{Théorème de Stone-Weierstrass dans le cas complexe}
\subsection{Notions de topologie nécessaires}



\section{Application aux séries de Fourier}


\section{Preuve constructive à l'aide des polynômes de Bernstein}

\clearpage
\printbibliography


\end{document}
