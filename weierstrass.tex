\documentclass{amsart}
\usepackage[T1]{fontenc} % French fontenc
\usepackage[french]{babel} % Use French as default lang
\usepackage{mathtools} % Usr amsmath commands 
%\usepackage{amssymb} % For \restriction command and math fonts, already in mathtools
\usepackage{enumitem} % Use Roman numbers in item enumeration
\usepackage[backend=biber]{biblatex} % Simple bibliography usage with biber
\addbibresource{weierstrass.bib}

\newtheorem{definition}[subsubsection]{Définition}
\newtheorem{theorem}[subsubsection]{Théorème}
\newtheorem{corollary}[subsubsection]{Corollaire}
\newtheorem{lemma}[subsubsection]{Lemme}

\title{AMS Théorème de Weierstrass et variations}
\author{Samy Amara et Guillaume Salloum}
%\email{guillaume.salloum@alumni-univ-avignon.fr}

\begin{document}
\begin{abstract}
	Nous nous intéresserons au théorème d'approximation de Weierstrass et
	à des variations utiles dans le cadre des séries de Fourier. 
	Dans un premier temps nous présenterons la version complexe 
	du thoérème de Stone-Weierstrass.
	Ensuite nous reprendrons le cas
	des séries de Fourier obtenu à l'aide du théorème de Fejer vu en cours. 
	Puis nous développerons un résultat sur les polynômes de Bernstein permettant
	de fournir une autre preuve constructive du théorème dans le cas réel.
\end{abstract}

\maketitle
\setcounter{tocdepth}{2}
\tableofcontents

\section{Théorème de Stone-Weierstrass dans le cas complexe}
\subsection{Dans les cadre d'un espace topologique séparé}

\par Dans cette partie nous traiterons le cas général du théorème où nous nous plaçons dans 
un espace topologique séparé et compact. Pour cela nous développons brièvement quelques notions préliminaires pour énoncer le théorème.
Afin de rendre la présentation plus proche du cours de \textit{Topologie et Analyse Hilbertienne}, cette approche n'est développée que dans le présent document écrit, pour la présentation orale nous adaptons systématiquement les résultats dans le cadre des espaces vectoriels normés étudiés en cours.

\begin{definition}[Espace séparé]
	Soit \( X \) un espace topologique. X est dit \textit{séparé}, ou \textit{de Hausdorff} s'il respecte l'axiome \( T_{2} \) : pour tout points \( x \) et \( y \) dans \( X \), il existe un voisinnage \( U \) de \( x \) et \( V \) de \( y \) tels que \( U \) et \( V \) sont disjoints.
\end{definition}

En particulier tout les espaces vectoriels normés sont issus d'espaces métriques, et sont donc séparés.

\begin{definition}[Algèbre, \cite{atiyah1969introduction}]
	Soit \( A \), \( B \) deux anneaux et \( f : A \rightarrow B \) 
	un morphisme d'anneaux. Si \( B \) est muni d'une structure de \textit{A-module}, alors B est une \textit{A-algèbre}.
\end{definition}

\begin{definition}[Sous-algèbre]
	Soit \( B \) une \textit{A-algèbre}, et \( C \subseteq B \). 
	\( C \) est une \textit{sous-algèbre de B} si 
	\( {f|}_C \) est un morphisme d'anneaux 
	et \( C \) possède une structure de \textit{A-module}. 
	En d'autres termes, \( C \) est un \textit{sous-module} de \( B \) 
	vu comme un \( A-module \).
\end{definition}

Ici nous nous intéressons à l'algèbre \( \mathcal{C} (X,\mathbb{C}) \) des fonctions continues de \( X \) dans \( \mathbb{C} \).


\subsection{Dans le cadre d'un espace vectoriel normé}

D'après le cours de \textit{Topologie et Analyse Hilbertienne}, nous avons vu que pour 
\((E, {\left\lVert . \right\rVert})\) un espace vectoriel normé et \( K \subseteq E \) compacte, alors 
\( \mathcal{C}(K,\mathbb{R}) \) l'ensemble des fonctions continues de \( K \) dans \( \mathbb{R} \) a une structure d'algèbre sur \( \mathbb{R} \). 
Ensuite, \( A \subset \mathcal{C}(K,\mathbb{R}) \) est une sous-algèbre si elle est stable 
pour les opérations définies sur \( \mathcal{C}(K,\mathbb{R}) \).
De plus, en considérant la norme de la convergence uniforme sur \( \mathcal{C}(K,\mathbb{R}) \) 
définie par \( {\left\lVert f \right\rVert_{\infty}} = \sup_{x \in K}{\left\lvert f(x) \right\rvert} \),
alors \(\mathcal{C}(K,\mathbb{R}) \) est une algèbre de Banach.
Enfin, on dit que \( A \subset \mathcal{C}(K,\mathbb{R}) \) sépare les points de \( K \) si pour \( x \neq y \) dans \( K \), alors il existe une fonction \( f \in A \) telle que 
\( f(x) \neq f(y) \). Avec ces outils, nous pouvons désormais énoncer le théorème.

\begin{theorem}[Stone-Weierstrass sur \( \mathbb{R} \)]
	Soit \( (E,{\left\lVert . \right\rVert}) \), \( K \subseteq E \) compacte et 
	\( A \subseteq \mathcal{C}(K,\mathbb{R}) \) une sous-algèbre vérifiant :
	\begin{enumerate}[label = (\roman*)]
		\item A contient les constantes,
		\item A sépare les points,
		\item \( \overline{A} = \mathcal{C}(K,\mathbb{R}) \)
	\end{enumerate}
	Alors toute fonction \( f : K \rightarrow \mathbb{R} \) est limite d'une suite de \( A \).
\end{theorem}

\par Dans le cas où l'on se place sur \( \mathbb{C} \),  





\section{Application aux séries de Fourier}


\section{Preuve constructive à l'aide des polynômes de Bernstein}

\clearpage
\printbibliography


\end{document}
